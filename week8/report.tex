\documentclass{article}
\usepackage[utf8]{inputenc}
\usepackage[english]{babel}
 
\usepackage{minted}
\usepackage{amsmath}
 
\title{Report for assignment 5}
\begin{document}

\section{Assignment 5}
\subsection{Python implementation of the heat equation}
In the first part of the assignment, I only used a list ,which is a native datastructure in python and may contain items from different types, and nested loops. The implementation was quite easy. However when I increase the size ($M\times N$)  of a rectangle or \textit{t1} it takes more time to be solved. Python does not use the contiguous memory for lists, so accessing to each element in a list become costly for the processor. 
\par

Based on the setup from the assignment's description my implementation has been executed in 54 seconds.

\subsection{NumPy and C implementations}
\subsubsection{Numpy}
The second implementation was done using the Numpy module. Converting the nested loops from the previous section and using a new syntax can solve a heat equation much faster. However, after I managed to convert the plain solver to the numpy version, the result was simpler than I would have expected.In the Numpy version I only used one loop which runs between 
[\textit{t0},\textit{t1}].

\begin{minted}{python}
loopCounter=t0
firstRow=1
firstCol=1
lastRow=n-1       
lastCol=m-1
while(loopCounter<=t1):
    u[firstRow:lastRow,firstCol:lastCol]=u[firstRow:lastRow,firstCol:lastCol] + dt \
    * (nu*u[firstRow-1:lastRow-1,firstCol:lastCol] \
    + nu*u[firstRow:lastRow,firstCol-1:lastCol-1] \
    - 4*nu*u[firstRow:lastRow,firstCol:lastCol] \
    + nu*u[firstRow:lastRow,firstCol+1:lastCol+1] \
    + nu*u[firstRow+1:lastRow+1,firstCol:lastCol] \
    + f[firstRow:lastRow,firstCol:lastCol])  
    loopCounter+=dt
return u
\end{minted}

\par
The numpy solver can work out the heat equation in 2.06 seconds. This is because the numpy arrays are saved using contiguous memory allocation. 
\subsubsection{Instant}
In this version I defined a function in pure C language and called that via Instant module. As a result it took only 289 ms to execute the function. This is why I decided not to use any python datatypes for my nested loop.
\par
The downside is that debugging the code is more difficult. However, it can be more useful when we only want to rewrite small piece of code in C.
\subsection{Testing}
For the testing part, I have used the py.test to solve the heat equation via the numpy solver. In addition, I have calculated the value of the analytic\textunderscore u and f in the nested loops structure. The error value for the ($100\times 50$) rectangle is less than 0.0012 and when I increase the size the error will decrease accordingly.
\subsection{User interface}

All functions that I have created for the user interface are included in heat\textunderscore equation\textunderscore ui.py. I have defined \textit{timefunc} funcion that is a wrapper for calling a function with the \textit{timeit} module.If we want to  measure time for a function with non-primitive arguments, it becomes necessary to call it within a wrapper function.

\par
I also defined two extra functions ,writeOnDisk and loadFromDisk, that are responsible for loading an initial matrix and saving the result on the disk.

\par
I have used the if-else statement in my function for checking a input parameters,optional and non-optional, which caused a little mess in my code. 
\end{document}