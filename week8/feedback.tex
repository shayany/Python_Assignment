\documentclass[a4paper]{article}

% Import some useful packages
\usepackage[margin=0.5in]{geometry} % narrow margins
\usepackage[utf8]{inputenc}
\usepackage[english]{babel}
\usepackage{hyperref}
\usepackage{minted}
\usepackage{amsmath}
\usepackage{xcolor}

\definecolor{LightGray}{gray}{0.95}

\title{Peer-review of assignment 5 for \textit{INF3331-ReviewedRepo}}
\author{Reviewer 1, Git-repo INF3331-Name1, {name1@email.no} \\
 		Reviewer 2, Git-repo INF3331-Name2, {name2@email.no} \\
		Reviewer 3, Git-repo INF3331-Name3, {name3@email.no}}
\date{Deadline: Tuesday, 10. November 2015, 23:59:59.}

\begin{document}
\maketitle

\section{Introduction and review guidelines}

\subsection{Assignment review}\label{sec:general_review}

The review should provide detailed feedback on the solution to the student. The main goal is to \emph{give constructive feedback and advice} on how to improve the solution. For each (coding) exercise, one should review the following points:

\begin{itemize}
  \item Is the code \textbf{working as expected}? For non-internal functions (in particular for scripts that are run from the command-line), does the program handle invalid inputs sensibly?
  \item Is the code written in \textbf{Pythonic way} \footnote{https://www.python.org/dev/peps/pep-0020/}? Is the code easy to read? Are the variable/class/function names sensible? Do you find overuse of classes or not sufficient use of functions or classes? Are there parts of the program that are hard to understand? 
  \item Is the code \textbf{well documented}? Are there docstrings?
  \item Can you find \textbf{unnecessarily complicated parts} of the program? If so, suggest an improved implementation.
  \item List the programming parts that are not answered.
\end{itemize}
You should use (shortened) code snippets where appropriate to show how to improve the solution. 


\subsection{Useful Latex snippets}
Here are some sample usage of Latex.

\noindent
Equation:
\begin{align}
2 \pi > 6
\end{align}
Sample code:

\begin{minted}[bgcolor=LightGray, linenos, fontsize=\footnotesize]{python}
import sys
print "This is a sample code"
sys.exit(0)
\end{minted}

\section{Review}\label{sec:review}

Specify the system (Python version, operating system, ...) that was used for the review.

%%%%%%%%%%%%%%%%%%%%%%%%%%%%%%%%%%%%%%%%%%%%%%%%%%%%%%%%%%
\subsection*{Assignment 5.1:  Python implementation of the heat equation}
Add a review based on section \ref{sec:general_review}.


%%%%%%%%%%%%%%%%%%%%%%%%%%%%%%%%%%%%%%%%%%%%%%%%%%%%%%%%%%
\subsection*{Assignment 5.2: NumPy and C implementations} \label{sec:assignment5.2}
Add a review for \underline{the numpy version and each C-version} in the repository based on section \ref{sec:general_review}.

In addition, review the following assignment specific items: 
\begin{itemize}
  \item Does the numpy solution make use of vectorisation sufficiently? 
  \item Does the solution make use of the C implementation sufficiently? 
  \item Do you have any suggestions on how to improve the solution for speed? E.g. can copying of data arrays be avoided.
\end{itemize}


%%%%%%%%%%%%%%%%%%%%%%%%%%%%%%%%%%%%%%%%%%%%%%%%%%%%%%%%%%
\subsection*{Assignment 5.3: Testing}
Add a review based on section \ref{sec:general_review}.


%%%%%%%%%%%%%%%%%%%%%%%%%%%%%%%%%%%%%%%%%%%%%%%%%%%%%%%%%%
\subsection*{Assignment 5.4:  Develop a user interface}
Add a review based on section \ref{sec:general_review}.


%%%%%%%%%%%%%%%%%%%%%%%%%%%%%%%%%%%%%%%%%%%%%%%%%%%%%%%%%%
\subsection*{Assignment 5.5: Latex report}
Review the following items: 
\begin{itemize}
\item Does the report describe the performed work well? 
\item Does the report have a comparison between the different implementations? 
\item Does the report explain reason for the runtime differences? Does it provide a technical understanding for performance of the different solutions?
\end{itemize}

%%%%%%%%%%%%%%%%%%%%%%%%%%%%%%%%%%%%%%%%%%%%%%%%%%%%%%%%%%
\subsection*{Assignment 5.6: More C-interfaces}
You can leave this section empty. The review for all C-interfaces should go to section 5.2.

%%%%%%%%%%%%%%%%%%%%%%%%%%%%%%%%%%%%%%%%%%%%%%%%%%%%%%%%%%
\subsection*{Assignment 5.7: Github activity plot}
Add a review based on section \ref{sec:general_review}.

\bibliographystyle{plain}
\bibliography{literature}

\end{document}
\documentclass[a4paper]{article}

% Import some useful packages
\usepackage[margin=0.5in]{geometry} % narrow margins
\usepackage[utf8]{inputenc}
\usepackage[english]{babel}
\usepackage{hyperref}
\usepackage{minted}
\usepackage{amsmath}
\usepackage{xcolor}
\definecolor{LightGray}{gray}{0.95}

\title{Peer-review of assignment 5 for \textit{INF3331-Shayan}}
\author{Thomas Fossøy Lyseggen, Git-repo INF3331-ThomasFossoy, {Thomaslyseggen@gmail.com} \\
 		Matthias Wenger, Git-repo INF3331-Matthias, {matthiw@student.matnat.uio.no} \\
		Sean Christian Dutch, Git-repo INF3331-Sean, {seancd@student.matnat.uio.no}}
\date{Deadline: Thursday, 12. November 2015, 23:59:59.}

\begin{document}
\maketitle

\section{Review}\label{sec:review}

platform linux2 -- Python 2.7.10 

%%%%%%%%%%%%%%%%%%%%%%%%%%%%%%%%%%%%%%%%%%%%%%%%%%%%%%%%%%
\subsection*{Assignment 5.1:  Python implementation of the heat equation}

Good implementation, works as expected. 

%%%%%%%%%%%%%%%%%%%%%%%%%%%%%%%%%%%%%%%%%%%%%%%%%%%%%%%%%%
\subsection*{Assignment 5.2: NumPy and C implementations} \label{sec:assignment5.2}

\subsubsection*{NumPy:}

It works as expected. But it is not really necassary to say \emph{firstRow=1} etc. Looks nice, but is somewhat superfluous.

\subsubsection*{C:}
The instructions you gave for setting up swig and instant, did not work. We tried it on 3 different computers. Cython works perfectly. Nice implementation.

%%%%%%%%%%%%%%%%%%%%%%%%%%%%%%%%%%%%%%%%%%%%%%%%%%%%%%%%%%
\subsection*{Assignment 5.3: Testing}

Works great. Since you set the default values, you don't need to pass them to the function.

%%%%%%%%%%%%%%%%%%%%%%%%%%%%%%%%%%%%%%%%%%%%%%%%%%%%%%%%%%
\subsection*{Assignment 5.4:  Develop a user interface}
You should have set the default values in the argparser. Makes it easier to read. You could do:
\begin{verbatim}
parser.add_argument("--t0", help="specify the start-time",default=0,type=float)
\end{verbatim}
And just pass \emph{args.t0} etc to the function instead of creating local variables.  

The verbosity mode doesn't work at all. It seems like it's not implemented. Some of the functionalities does not work together, like \emph{timeit} and \emph{save plot}.

%%%%%%%%%%%%%%%%%%%%%%%%%%%%%%%%%%%%%%%%%%%%%%%%%%%%%%%%%%
\subsection*{Assignment 5.5: Latex report}
The report is well written, and you show technical understanding for performance of the different solutions. You haven't done any comparisons between the different implementations.

%%%%%%%%%%%%%%%%%%%%%%%%%%%%%%%%%%%%%%%%%%%%%%%%%%%%%%%%%%
\subsection*{Assignment 5.6: More C-interfaces}
See subsection 5.2. 

%%%%%%%%%%%%%%%%%%%%%%%%%%%%%%%%%%%%%%%%%%%%%%%%%%%%%%%%%%
\subsection*{Assignment 5.7: Github activity plot}
Works as expected. Nice looking plot. 

%%%%%%%%%%%%%%%%%%%%%%%%%%%%%%%%%%%%%%%%%%%%%%%%%%%%%%%%%%
\subsection*{Summary}
Generally a good assignment, but could not get swig and instant to work. The instructions given did not work, and you should have written it in the report as well. The documentation could be more in depth. You have decided to run all the different solutions in one file (main), which is ok had all your implementations worked smoothly. Unfortunately seeing as swig etc. were faulty it made it harder to see whether the other implentations worked aswell.

\bibliographystyle{plain}
\bibliography{literature}

\end{document}
